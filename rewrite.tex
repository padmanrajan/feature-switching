\documentclass{article}
\usepackage[dvips]{graphicx, epsfig}
\usepackage{subfigure}
\usepackage{amssymb,amsmath}

\title{Feature-switching paper draft today}
\begin{document}

\section{Introduction}
Feature extraction is a crucial step in pattern recognition systems. For
speech signals, feature extraction is a transformation from the acoustic space
to a feature space. In text-independent speaker verification, the objective is
to determine if two utterances (the train utterance and the test utterance) are
both spoken by a particular speaker. We expect that the transformation into the
feature space effectively discriminates the utterances spoken by the speaker
under consideration with those spoken by other speakers. Most speaker
verification systems, however, apply the same transformation, no matter which
speaker is being considered. In this paper, we explore a new paradigm, which
exploits the \emph{diversity of information} present in different feature spaces
for speaker verification. The underlying assumption is that different speakers
may be better discrimated by different features. Hence, performance can
be improved by selecting the ``well-suited'' feature transformation for
each speaker. We term this technique \emph{feature-switching}. 
%Building of such
%a system involves determining the better-suited feature for every speaker.

Traditionally, the diversity of different feature transformations have been
utilised by combining them. These include the so-called \emph{early fusion},
which is a combination at the feature level, and \emph{late fusion}, which is at
the classifier (or decision) level. Combining the information from multiple
feature tranformations usually results in improved performance, though
with an increase in system complexity. Feature-switching aims to utilise
information from multiple feature representations, although in a non-traditional
manner. Early fusion systems typically work by concatenating feature vectors;
hence the resulting feature space is of higher dimensons. This in turn requires
more training data to effectively train statistical models. Late fusion requires
individual systems to be trained and fused; in platforms with limited memory or
storage space, this could be undesirable. The feature-switching technique
attempts to get the benefit of multiple feature representations, at the same
time, reduce the system complexity and storage requirements.

Most feature representations transform the speech signal into its spectral
representation. The short-term Fourier transform is a complex quantity, with
information present in both magnitude and phase spectra. It is known from linear
system theory, that non minimum-phase signals have different information in
magnitude and phase spectra \cite{oppenheim}. Several studies \cite{ph1, ph2}
have shown the complementarity of magnitude and phase, and how combining feature
vectors derived from each of them improves performance in various tasks. In this
paper, we study the effectiveness of feature-switching for speaker verification
using feature representations from magnitude-based and phase-based features. We
perform feature-switching using the standard Mel-frequency cepstra (MFCC)
\cite{mfcc}, which
is derived from short-term magnitude, and the modified group delay feature
(MODGDF) \cite{hegdeModgdf}, which is derived from short-term phase. For each speaker, the
better-suited of these two representations is determined beforehand. Then,
feature-switching is applied for speaker verification by verifying some speakers
using MFCC features, and others using MODGDF features.

We study feature-switching for speaker verification in the context of the
classical GMM-UBM system \cite{reynoldsAdaptedGMM}, and also the more
sophisticated i-vector based representation \cite{dehak_ivector}. In both cases,
our studies show that feature-switching improves verification accuracy, when compared to
conventional baseline systems, as well as fusion systems. Experiments are performed on the
NIST 2010 speaker recognition evaluation (SRE) \cite{nist2010SRE} data.

Earlier papers interspeech, asha

\section{Separability analysis in different feature spaces}
The underlying hypothesis for feature-switching is that speakers are separated
differently in different feature spaces. To study this in more detail, we
perform separability studies in MFCC space and MODGDF space.

In the classical GMM-UBM framework \cite{reynoldsAdaptedGMM}, a speaker is
represented by a Gaussian mixture model (GMM). Given feature vectors extracted
from a speech utterance, the likelihood ratio of the speaker GMM and that of the
universal background model (UBM) is computed. Better separation between the GMM
and the UBM implies improved accuracy in verification.

Figures \ref{fig:gmm_mgd_opt} and \ref{fig:gmm_mfc_opt} illustrate the
separability obtained in MFCC and MODGDF feature spaces for two different
speakers. The mean vectors of a 32-mixture GMM and UBM are plotted in
two-dimensional space. In these figures, the 39-dimensional MFCC and MODGDF
feature vectors are reduced to two dimensions using the Sammon mapping technique
\cite{sammon}. It can be seen that in Figure \ref{fig:gmm_mgd_opt}, there is
better separation between the GMM and the UBM in the MODGDF space, when compared
to the MFCC space. Thus, it is likely that this speaker is better discriminated
against the UBM in the MODGDF space. Similarly, for a different speaker, 
Figure \ref{fig:gmm_mfc_opt} shows better separation in the MFCC space. Thus,
for this speaker, performing verification in the MFCC space gives better
discrimination between the UBM and the GMM. 

Similar anaysis is performed in ivector space.


\begin{figure}[!h]
\centering
\subfigure[Speaker and UBM centriods in 2 dimensions in MFCC space. 
There is high overlap and hence higher confusablity in this space.]{
\includegraphics[scale=0.30]{figures/partA_highOverlap.eps}
\label{fig:subfig1}
}
\subfigure[Speaker and UBM centriods in MODGDF space. There is a distinct 
clustering of centriods which is shown by the SVM boundary.]{
\includegraphics[scale=0.30]{figures/partC_lessOverlap.eps}
\label{fig:subfig2}
}
\caption[]{Subfigures \subref{fig:subfig1} and \subref{fig:subfig2} 
show the same speaker and background model centriods in MFCC and 
MODGDF spaces. This speaker and the UBM are better separable in 
MODGDF space.}
\label{fig:gmm_mgd_opt}
\end{figure}

\begin{figure}[h]
\centering
\subfigure[Speaker and UBM centriods in 2 dimensions in MODGDF space. There 
is high overlap and hence higher confusablity in this space.]{
\includegraphics[scale=0.30]{figures/partB_highOverlap.eps}
\label{fig:subfig3}
}
\subfigure[Speaker and UBM centriods in MFCC space. There is a distinct 
clustering of centriods which is shown by the SVM boundary.]{
\includegraphics[scale=0.30]{figures/partD_lessOverlap1.eps}
\label{fig:subfig4}
}
\caption[]{Subfigures \subref{fig:subfig3} and \subref{fig:subfig4} show the 
same speaker and background model centriods in MFCC and MODGDF spaces. This 
speaker and the UBM are better separable in MFCC space.}
\label{fig:gmm_mfc_opt}
\end{figure}





\section{Optimal feature selection and feature-switching}

Speaker verification is a two-class problem. A verification trial consists of a
test utterance from an unknown speaker, and a speaker claim. Feature-swtching
can be naturally applied to the verification scenario by performing
verification in the well-suited feature space of the claimed speaker. This
well-suited feature representation is henceforth termed as the \emph{optimal
feature}. The overall architecture of the feature-switching system is shown in
Figure \ref{fig:systemArch}. The optimal feature is determined for every speaker
during enrollment, and stored into a lookup-table. During testing, the optimal
feature of the claimed speaker is looked-up, and verification is performed in
the optimal feature space.

\begin{figure}[th]
\centering
\includegraphics[width=0.5\textwidth]{figures/blank.eps}
\caption{System architecture for feature-switching.}
\label{fig:systemArch}
\end{figure}



\subsection{Determining the optimal feature for the GMM-UBM framework}

For the GMM-UBM framework, the method of determining the optimal feature for a
particular speaker, given a set of candidate feature representations, is
described in \cite{padmanInterspeech2010}. The optimal feature is determined by
evaluating the representation ability and discrimination ability
of each candidate feature representation. Given an enrollment utterance, the mutual information
between extracted feature vectors and the complex Fourier transform (CFT) is used as an
estimate of information captured by the feature vectors. Thus, the
representation ability of the feature representation is given as
\begin{equation}
\textrm{mi}(\textrm{CFT},X),
\end{equation}
where $\textrm{mi}$ represents the mutual information, CFT is the complex
Fourier transform, and $X$ is a set of feature vectors, which are computed from
the utterance.




The discrimination ability
is determined by estimating the Kullback-Leibler divergence (KL-divergence)
between the UBM and the speaker GMM adapted from it. Because of the one-to-one
correspondence between the mixture components of the background model and the
speaker model, the KL-divergence can be expressed in closed-form. 
For two unimodal Gaussian distributions $\hat{f}$ and $\hat{g}$, the KL-divergence has the closed form
expression
\begin{equation}
\begin{split}
\textrm{kld}(\hat{f},\hat{g}) = \frac{1}{2}\left[ \log \frac{|\Sigma_g|}{|\Sigma_f|} +
	\textrm{Tr}|\Sigma^{-1}_g\Sigma_f| - d + \right. \\ 
\left. (\mu_f-\mu_g)^T\Sigma_g^{-1}(\mu_f-\mu_g) \frac{}{} \right], 
\end{split}
\label{eq:kldGaussians}
\end{equation}

where $\hat{f} = \mathcal{N}(\mu_f,\Sigma_f)$ and $\hat{g} = \mathcal{N}(\mu_g,\Sigma_g)$.

For speaker models whose means are adapted from the UBM (the 
covariances and mixture weights are same as that of the UBM), the KL-divergence
reduces to 
\begin{equation}
\textrm{kld}(\lambda_{\textrm{spk}},\lambda_{\textrm{UBM}}) = 
	\Sigma_i\, \pi_i\, \textrm{kld}(f_i,g_i),
\label{eq:gmmAdaptedKLD}
\end{equation}

%where $\lambda_{\textrm{spk}} = \Sigma_i \, \pi_i \, \mathcal{N}(\mu_i^{\textrm{spk}},\Sigma)$ 

where $\lambda_{\textrm{spk}} = \Sigma_i \, \pi_i \, f_i$, and
$\lambda_{\textrm{UBM}} = \Sigma_i \, \pi_i \, g_i$, and
$f = \mathcal{N}(\mu_{\textrm{spk}},\Sigma)$, and
$g = \mathcal{N}(\mu_{\textrm{UBM}},\Sigma)$ and
$\pi$ are the mixture weights.


The optimal feature for a particular speaker is determined from the combined
representative and discriminative measures of each of the $P$ candidate
features. For the $p$th feature representation, we determine

\begin{eqnarray*}
&& \theta_p = \textrm{mi}(\textrm{CFT},X_{p}) \\ && \gamma_p =
\textrm{kld}(\lambda_{\textrm{spk},p},\lambda_{\textrm{UBM},p}) \\
\end{eqnarray*} 
where $X_p$ are feature vectors, the speaker model $\lambda_{\textrm{spk}}$ and 
UBM $\lambda_{\textrm{UBM}}$ are in the $p$th feature space, and $p$ ranges from 1 to $P$.


%\in
%\{\textrm{MFCC},\textrm{LPCC},\textrm{MODGDF},\textrm{fSlope}\}$,  $\mathcal{X}$
%represents the complex Fourier spectrum, $\mathcal{Y}_i$ represents the $i$ th feature
%representation, $\lambda_{\textrm{spk},i}$ is the speaker model and $\lambda_{\textrm{UBM},i}$
%is the background model,
%using the $i$ th feature representation.


A linear combination of these two measures are used to determine the optimal
feature

\begin{equation}
\phi_p = \alpha \theta_p  + (1-\alpha) \gamma_p
\end{equation}

where $\alpha$ is a weighting parameter determined experimentally. The optimal
feature $\hat{p}$ is selected as 

\begin{equation}
\hat{p} = \arg\max_p \{\phi_p\}
\label{eq:optFeat}
\end{equation}

\subsection{Determining the optimal feature in the i-vector framework}

The i-vector representation \cite{dehak_ivector} is a a fixed-length
representation of speech utterances, which usually have a variable number of
traditional feature vectors.  Given an $FM \times 1$ supervector of means $\mu$
derived from a UBM, a speaker and recording specific supervector $s$ is assumed to of
the form

\begin{equation}
s = \mu + T w.
\end{equation}

Here, acoustic feature vector is $F$-dimensional, the UBM has $M$ components,
$T$ is an $FM \times D$ low-rank matrix, and $w$ is a $D \times 1$ latent
vector, with a standard normal distribution $w \sim \mathcal{N}(0,I)$. The
i-vector is estimated as the mean of the posterior distribution of $w$, given
the utterance. Procedures to estimate the hyperparameters $(\mu, T)$ and 
estimate i-vectors from an utterance can be found in \cite{dehak_ivector}.

The i-vector representing an utterance encodes information about the message,
the speaker, and the channel. To compensate for unwanted channel effects,
several preprocessing steps like length normalization \cite{garcia_lengthNorm},
and within-class covariance normalization (WCCN) \cite{wccn} is performed. A
popular method to measure similarity between two i-vectors is cosine 
distance \cite{dehak_ivector}.

For a given utterance, i-vectors can be estimated from different acoustic
feature vectors and their associated hyperparameters. Hence, the
better-suited ivector representation for a particular speaker can be estimated
from amongst i-vectors extracted from different acoustic features. One way of
doing this is by determining the i-vector representation which has the maximum
distance between the given speaker and the other enrollment speakers. If there
are $N$ speakers, the optimal i-vector representation $\Hat{\Hat{p}}$ 
for the $i$th speaker can be determined as

\begin{equation}
\Hat{\Hat{p}} = \arg\max_p \{s_p\}
\end{equation}

where 

\begin{equation}
s_p = \frac{\displaystyle \sum_{j=1, \; i \neq j}^N d(w_{p,i},w_{p,j})}{N}
\end{equation}

Here, $w_{p,j}$ represents the enrollment i-vector for the $j$th speaker
extracted using the $p$th feature representation. $d$ is a distance measure (for
example, cosine distance) between i-vectors. For the $i$th speaker, the average
distance with the other enrollment speakers is used to determine the optimal
i-vector representation.

\section{Feature extraction}

\section{Experimental evaluation}

We perform an experimental evaluation of the proposed feature-switching
mechanism, in the context of both the GMM-UBM framework, and the i-vector
framework. Equal error rate (EER) is used as the evaluation metric.

\subsection{Baseline verification systems}
\textbf{GMM-UBM system.} Give details here. UBM size, top C, etc.
The baseline systems are denoted as follows.
\begin{enumerate}
\item UBM-MFC-male and UBM-MFC-female
\item UBM-MGD-male and UBM-MGD-female
\end{enumerate}

\textbf{i-vector system.} Give details here. T matrix size, feature dimension,
WCCN details etc. 
\begin{enumerate}
\item ivec-MFC-male and ivec-MFC-female
\item ivec-MGD-male and ivec-MGD-female
\end{enumerate}

\subsection{Datasets used}
The evaluation is performed using the telephone conditions of the NIST 2010
speaker recognition evaluation (SRE.) These are the following NIST SRE 2010
conditions: C5, C6, C7, C8, C9 \cite{nist2010SRE}.
\textit{Put a table here.}

Give details of what data you used for extracting hyperparameters: UBM, Tmatrix
etc.


\subsection{Feature-switching framework}
In the proposed feature-switching framework, different speaker claims are
verified using different feature representations. The following
feature-switching systems are evaluated.
\begin{enumerate}
\item FS-UBM-male and FS-UBM-female
\item FS-ivec-male and FS-ivec-female
\end{enumerate}

\textit{Give details about the FS systems here}

When a test utterance and a speaker claim is input, the feature-switching system
obtains the optimal feature from the lookup table. Then the corresponding
baseline system (which is optimal for the claimed speaker) is used to perform the verification.

\textbf{GMM-UBM feature-switching system.}
How did you determne the $\alpha$ values?
How many speakers got MFCC and how many got MGD?

\textbf{i-vector feature-switching system.}
How many speakers got MFCC and how many got MGD?

Are the optimal features different for UBM-GMM and i-vector systems?









\bibliography{refs}
\bibliographystyle{IEEEtran}




%Speaker identification is an $N$-class problem, whereas speaker verification

\end{document}
