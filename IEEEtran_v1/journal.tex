\documentclass[preprint,12pt,5p]{elsarticle}


\usepackage{graphicx}
\DeclareGraphicsExtensions{.eps}
\usepackage{epsfig}
\usepackage{amssymb,amsmath,bm}
\usepackage{textcomp}
\usepackage{caption}
\usepackage{multirow}
\usepackage{nonfloat}
\usepackage{algorithm}
\usepackage{subcaption}
\usepackage{algorithmic}
\usepackage{lipsum}
\usepackage{amsmath}
\usepackage{array}
\usepackage[caption=false,font=footnotesize,labelfont=sf,textfont=sf]{subfig}
\usepackage{url}
\usepackage{nonfloat}
\usepackage{flushend}

\DeclareMathSizes{10}{10}{10}{10}

\begin{document}


\begin{frontmatter}

\title{ {\bf Feature-switching : Dynamic feature selection for speaker verification}}

\author[label1]{Saranya M S}
\address[label1]{Indian Institue of Technology Madras}

\ead{saranms@cse.iitm.ac.in}
%\ead[url]{author-one-homepage.com}

\author[label2]{Padmanabhan R}
\address[label2]{Indian Institute of Technology Mandi}
\ead{padman@iitmandi.ac.in }

\author[label1]{Hema A Murthy}
\ead{hema@cse.iitm.ac.in}

\begin{abstract}

Conventional speaker verification systems utilize information from different
feature representations by means of fusion. In this paper, we propose an
alternative technique which achieves similar effect, but 
utilizes a more effective feature selection. The underlying assumption of the 
method is that different speakers may be better represented, and hence better 
verified, in different feature spaces.  This technique, which we term as feature-switching, 
performs verification using a feature representation, most suitable to the
speaker under consideration. %A set of candidate feature representations are defined. 
Out of a possible set of candidate representations, the most optimal
representation of a speaker is determined during training. Then verification is
performed using the optimal feature of the claimed speaker. Experimental evaluation
of feature-switching is performed utilizing the classical GMM-UBM speaker
verification system, as well as the i-vector verification system. Our results show
that feature-switching achieves improved performance compared to conventional as well
as fusion based-systems.

\end{abstract}

\begin{keyword}
feature switching \sep i-vector \sep speaker verification \sep UBM-GMM \sep Total variability Space (TVS)\sep MODGD
\end{keyword}

\end{frontmatter}

\flushend 

\section{Introduction}
\label{sec:intro}
Feature extraction is an important step in pattern recognition systems. For
speech signals, feature extraction is a transformation from the acoustic space
to a feature space. In text-independent speaker verification, the objective is
to determine if two utterances (the enrollment utterance and the test utterance) are
both spoken by a particular speaker. We expect that, the transformation into the
feature space effectively discriminates the utterances spoken by the speaker
under consideration with those spoken by other speakers. Most speaker
verification systems, however, apply the same transformation, no matter which
speaker is being considered. In this paper, we explore a new paradigm, which
exploits the \emph{diversity of information} present in different feature spaces
for speaker verification. The underlying assumption is that different speakers
may be better discriminated in different feature spaces. Hence, performance can
be improved by utilizing the `well-suited' feature transformation for
each speaker. We term this technique \emph{feature-switching}. 


Traditionally, the diversity of different feature transformations have been
utilized by combining them. These include the so-called \emph{early fusion},
which is a combination at the feature level, and \emph{late fusion}, which is at
the classifier (or decision) level. Combining the information from multiple
feature transformations usually results in improved performance, albeit
with an increase in system complexity. Feature-switching aims to utilize
information from multiple feature representations, in an unconventional manner. 
Early fusion systems typically work by concatenating feature vectors;
hence the resulting feature space is of higher dimensions. This in turn requires
more data to effectively train statistical models. Late fusion requires
individual systems to be developed and fused; in platforms with limited
processing or storage space, this could be undesirable. The feature-switching
technique attempts to get the benefit of multiple feature representations, at the
same time, reduce the system complexity and storage requirements.

Most feature representations transform the speech signal into its spectral
representation. The short-term Fourier transform is a complex quantity, with
information present in both magnitude and phase spectra. It is known from linear
system theory, that non minimum-phase signals have different information in
magnitude and phase spectra \cite{oppenheim}. Several studies \cite{complement2}
have shown the complementarity of magnitude and phase, and how combining feature
vectors derived from each of them improves performance in various tasks. In this
paper, we study the effectiveness of feature-switching for speaker verification
using feature representations from magnitude-based and phase-based features. We
perform feature-switching using the standard Mel-frequency cepstra (MFCC)
\cite{mfcc}, which is derived from short-term magnitude, and the modified group 
delay feature (MODGDF) \cite{hegdeModgdf}, which is derived from short-term phase.
For each speaker, the better-suited of these two representations is determined
beforehand. Then, feature-switching is applied for speaker verification by verifying
some speakers using MFCC features, and others using MODGDF features.

We study feature-switching for speaker verification in the context of the
classical GMM-UBM system \cite{reynoldsAdaptedGMM}, and also the more
sophisticated i-vector based representation \cite{dehak_ivector}. In both cases,
our studies show that feature-switching improves verification accuracy, when
compared to conventional systems, which use only a single feature
representation. In addition, feature-switching also shows improvement over
fusion systems. Experiments are performed on the NIST 2010 speaker recognition
evaluation (SRE) \cite{nist2010SRE} data.

The rest of the paper is organized as follows: Separation of features in different feature spaces is analyzed in Section~\ref{sec:separability}. The process of selecting the optimal feature and feature switching is explained in Section~\ref{sec:optFeat}. The candidate features used for feature selection is explained in Section~\ref{sec:featExt} followed by the explanation of experimental evaluation in Section~\ref{sec:expts} and conclusion in Section~\ref{sec:conclude}. The acknowledgement for this work is given in Section~\ref{sec:ack}

\section{Separability analysis in different feature spaces}
\label{sec:separability}
The underlying hypothesis for feature-switching is that speakers are separated
differently in different feature spaces. To study this in detail, we
perform separability studies in MFCC space and MODGDF space.

In the classical GMM-UBM framework \cite{reynoldsAdaptedGMM}, a speaker is
represented by a Gaussian mixture model (GMM). Given feature vectors extracted
from a speech utterance, the likelihood ratio of the speaker GMM and the
universal background model (UBM) is computed. Better separation between the GMM
and the UBM implies improved accuracy in verification.

\begin{figure}[h]
\centering 
\begin{minipage}[c]{0.5\textwidth}
\centering 
%    \includegraphics[width=0.99\textwidth]{../figures/tdcap_mfcc.eps}
    \includegraphics[scale=0.46]{../figures/tdcap_mfcc.eps}
	\caption*{(a)}
\end{minipage}

\begin{minipage}[c]{0.5\textwidth}
\centering  
%    \includegraphics[width=0.99\textwidth]{../figures/tdcap_mgd.eps}
\vspace{3mm}
    \includegraphics[scale=0.46]{../figures/tdcap_mgd.eps}

	\caption*{(b)}
\end{minipage}
\caption{Sub-figures (a) and (b) show a speaker and background model
centroids in MFCC and MODGDF spaces. This speaker and the UBM are better
separable in MODGDF space.}
\label{fig:ubm_sep1}
\end{figure}

\begin{figure}[h]
\centering 
\begin{minipage}[c]{0.5\textwidth}
\centering 
%    \includegraphics[width=0.99\textwidth]{../figures/tecer_mfcc.eps}
    \includegraphics[scale=0.46]{../figures/tecer_mfcc.eps}
	\caption*{(a)}
\end{minipage}

\begin{minipage}[c]{0.5\textwidth}
\centering  
%    \includegraphics[width=0.99\textwidth]{../figures/tecer_mgd.eps}
\vspace{3mm}
    \includegraphics[scale=0.46]{../figures/tecer_mgd.eps}
	\caption*{(b)}
\end{minipage}
\caption{Sub-figures (a) and (b) show a speaker and background model
centroids in MFCC and MODGDF spaces. This speaker and the UBM are better
separable in MFCC space.}
\label{fig:ubm_sep2}
\end{figure}

Figures \ref{fig:ubm_sep1} and \ref{fig:ubm_sep2} illustrate the
separability obtained in MFCC and MODGDF feature spaces for two 
speakers. The mean vectors of a 32-mixture GMM and UBM are plotted in
two-dimensional space. In these figures, the 39-dimensional MFCC and MODGDF
feature vectors are reduced to two dimensions using the Sammon mapping technique
\cite{sammon}. 

Sammon mapping represents high dimensional vectors in a 
lower dimensional space such that the geometric relations between the original data points
are preserved. This mapping technique facilitates the visualization of higher dimensional data
in the lower dimension. The measure used by Sammon mapping is designed to 
minimise the differences between corresponding inter-point distances in the two spaces \cite{sammon1}. 

It can be seen that in Figure \ref{fig:ubm_sep1}, there is  better separation
between the GMM and the UBM in the MODGDF space, when compared to the MFCC
space. Thus, it is likely that this speaker is better discriminated against the
UBM in the MODGDF space. Similarly, for another speaker, Figure
\ref{fig:ubm_sep2} shows better separation in the MFCC space. Thus, for this
speaker, performing verification in the MFCC space gives better discrimination
between the UBM and the GMM. 

\begin{figure}[h!tb]
\centering 
\begin{minipage}{0.5\textwidth}
\centering 
%\includegraphics[width=0.99\textwidth]{../figures/spkr_1_mfcc.eps}
\includegraphics[scale=0.46]{../figures/spkr_1_mfcc.eps}
\caption*{(a)}
\label{fig:subfig3}
\end{minipage}
\begin{minipage}{0.5\textwidth}
\centering 
\vspace{3mm}
%\includegraphics[width=0.99\textwidth]{../figures/spkr_1_mgd.eps}
\includegraphics[scale=0.46]{../figures/spkr_1_mgd.eps}
\caption*{(b)}
\label{fig:subfig4}
\end{minipage}
\caption{Sub-figures (a) and (b) show the i-vectors derived from MFCC
and MODGDF for target and impostor trials. For this speaker, target and impostor
i-vectors are better separable in MFCC space.}
\label{fig:ivec_separation1}
\end{figure}

\begin{figure}[h!tb]
\centering 
\begin{minipage}{0.5\textwidth}
\centering 
%\includegraphics[width=0.99\textwidth]{../figures/spkr_2_mfcc.eps}
\includegraphics[scale=0.46]{../figures/spkr_2_mfcc.eps}
\caption*{(a)}
\label{fig:subfig5}
\end{minipage}
\begin{minipage}{0.5\textwidth}
\centering 
\vspace{3mm}
%\includegraphics[width=0.99\textwidth]{../figures/spkr_2_mgd.eps}
\includegraphics[scale=0.46]{../figures/spkr_2_mgd.eps}
\caption*{(b)}
\label{fig:subfig6}
\end{minipage}
\caption{Sub-figures (a) and (b) show the i-vectors derived from MFCC
and MODGDF for target and impostor trials. For this speaker, target and impostor
i-vectors are better separable in MODGDF space.}
\label{fig:ivec_separation2}
\end{figure}

Similar analysis is performed in i-vector space by considering i-vectors derived
from different feature representations.  For a given speaker, target trials are
those spoken by the speaker himself or herself, and are also called true-speaker
trials. Non-target trials are spoken by other speakers, and are also called
imposter trials. In this figure, 38-dimensional dimension i-vectors are reduced
to two dimensions using Sammon mapping. The better separation of target and
non-target i-vectors in optimal feature  can be readily seen in Figure
\ref{fig:ivec_separation1} and \ref{fig:ivec_separation2}.  


\section{Optimal feature selection and feature-switching}
\label{sec:optFeat}

Speaker verification is a two-class problem. A verification trial consists of a
test utterance from an unknown speaker, and a speaker claim. Feature-switching
can be naturally applied to the verification scenario by performing
verification in the well-suited feature space of the claimed speaker. This
well-suited feature representation is henceforth termed as the \emph{optimal
feature}. The overall architecture of the feature-switching system is shown in the
Figure \ref{fig:systemArch}. The optimal feature is determined for every speaker
during enrollment, and stored in a lookup-table. During testing, the optimal
feature of the claimed speaker is looked-up, and verification is performed in
the optimal feature space.

\begin{figure}[th]
\centering
\includegraphics[scale=0.25]{../figures/FS_Architect.eps}
\caption{System architecture for training and testing phases in feature-switching.}
\label{fig:systemArch}
\end{figure}

\subsection{Determining the optimal feature for the GMM-UBM framework}
\label{subsec:ubm_optFeat}

For the GMM-UBM framework, the method of determining the optimal feature for a
particular speaker, given a set of candidate feature representations, was
described in \cite{padmanInterspeech2010}. The optimal feature is determined by
evaluating the representation ability and discrimination ability
of each candidate feature representation. Given an enrollment utterance, the
mutual information between extracted feature vectors and the complex Fourier
transform (CFT) is used as an estimate of information captured by the feature
vectors. Thus, the representation ability of the feature representation is given
as 
\begin{equation}
\textrm{mi}(\textrm{CFT},X),
\end{equation}
where $\textrm{mi}$ represents the mutual information, CFT is the complex
Fourier transform, and $X$ is a set of feature vectors, which are computed from
the utterance.

The discrimination ability is determined by estimating the Kullback-Leibler divergence
(KL-divergence) between the UBM ($\lambda_{\textrm{ubm}}$) and the speaker GMM ($\lambda_{\textrm{spk}}$)
adapted from it. Because of the one-to-one
correspondence between the mixture components of the background model and the speaker model, 
the KL-divergence can be expressed in closed-form. For two uni-modal Gaussian distributions
$\hat{f}$ and $\hat{g}$, the KL-divergence has the closed form expression
\begin{equation}
\begin{split}
\textrm{kld}(\hat{f},\hat{g}) = \frac{1}{2}\left[ \log \frac{|\Sigma_g|}{|\Sigma_f|} +
	\textrm{Tr}|\Sigma^{-1}_g\Sigma_f| - d + \right. \\ 
\left. (\mu_f-\mu_g)^T\Sigma_g^{-1}(\mu_f-\mu_g) \frac{}{} \right], 
\end{split}
\label{eq:kldGaussians}
\end{equation}

where $\hat{f} = \mathcal{N}(\mu_f,\Sigma_f)$ and $\hat{g} = \mathcal{N}(\mu_g,\Sigma_g)$.

For multi-modal speaker models $\lambda_{\textrm{spk}}$, whose means
$\mu_{\textrm{spk},i}$ are adapted from the means $\mu_{\textrm{ubm},i}$ of the UBM
model $\lambda_{\textrm{ubm}}$ (the covariances and mixture weights are same as
that of the UBM), the KL-divergence reduces to 
\begin{equation}
\textrm{kld}(\lambda_{\textrm{spk}},\lambda_{\textrm{ubm}}) = 
	\displaystyle \sum_i\ \pi_i\ \textrm{kld}(f_i,g_i),
\label{eq:gmmAdaptedKLD}
\end{equation}
%where $\lambda_{\textrm{spk}} = \Sigma_i \, \pi_i \, \mathcal{N}(\mu_i^{\textrm{spk}},\Sigma)$ 
where, \\
$\lambda_{\textrm{spk}} = \displaystyle \sum_i \, \pi_i \, f_i$, \\
$\lambda_{\textrm{ubm}} = \displaystyle\sum_i \, \pi_i \, g_i$, \\
$f_i = \mathcal{N}(\mu_{\textrm{spk},i},\Sigma_i)$, and \\
$g_i = \mathcal{N}(\mu_{\textrm{ubm},i},\Sigma_i)$.

\noindent $\pi$ are the mixture weights and
$i$ varies from 1 to $C$, the number of mixture components. Here, $f_i$ and $g_i$ are
corresponding unimodal Gaussian distributions.


The optimal feature for a particular speaker is determined from the combined
representative and discriminative measures of each of the $P$ candidate
features. For the $p$th feature representation, we determine

\begin{eqnarray*}
&& \theta_p = \textrm{mi}(\textrm{CFT},X_{p}), \\
&& \gamma_p =
\textrm{kld}(\lambda_{\textrm{spk},p},\lambda_{\textrm{ubm},p}), \\
\end{eqnarray*} 
where $X_p$ are feature vectors, the speaker model $\lambda_{\textrm{spk}}$ and 
UBM $\lambda_{\textrm{ubm}}$ are in the $p$th feature space, and $p$ ranges from 1 to $P$.


%\in
%\{\textrm{MFCC},\textrm{LPCC},\textrm{MODGDF},\textrm{fSlope}\}$,  $\mathcal{X}$
%represents the complex Fourier spectrum, $\mathcal{Y}_i$ represents the $i$ th feature
%representation, $\lambda_{\textrm{spk},i}$ is the speaker model and $\lambda_{\textrm{UBM},i}$
%is the background model,
%using the $i$ th feature representation.


A linear combination of these two measures is determined 

\begin{equation}
\phi_p = \alpha \theta_p  + (1-\alpha) \gamma_p,
\label{eq:phiFunc}
\end{equation}

where $\alpha$ is a weighting parameter determined experimentally. The optimal
feature $\hat{p}$ for a given speaker is determined as 

\begin{equation}
\hat{p} = \arg\max_p \{\phi_p\}.
\label{eq:ubm_optFeat}
\end{equation}

\subsection{Determining the optimal feature in the i-vector framework}
\label{subsec:ivec_optFeat}

The i-vector representation \cite{dehak_ivector} is a a fixed-length
representation of speech utterances, which usually have a variable number of
traditional feature vectors.  Given an $FM \times 1$ super-vector of means $\mu$
derived from a UBM, a speaker and recording specific super-vector $s$ is assumed to of
the form

\begin{equation}
s = \mu + T w.
\end{equation}

Here, the acoustic feature vector is $F$-dimensional, the UBM has $M$ components,
$T$ is an $FM \times D$ low-rank matrix, and $w$ is a $D \times 1$ latent
vector, with a standard normal distribution $w \sim \mathcal{N}(0,I)$. The
i-vector is estimated as the mean of the posterior distribution of $w$, given
the utterance. Procedures to estimate the hyper-parameters $\mu$ and $T$, and 
estimate i-vectors from an utterance, can be found in \cite{dehak_ivector}.

The i-vector representing an utterance encodes information about the message,
the speaker, and the channel. To compensate for unwanted channel effects,
several preprocessing steps like length normalization \cite{garcia_lengthNorm},
and within-class covariance normalization (WCCN) \cite{wccn} is performed. A
popular method to measure similarity between two i-vectors is by computing the
cosine distance \cite{dehak_ivector}.

For a given utterance, i-vectors can be estimated from different acoustic
feature vectors and their associated hyper-parameters. Hence, the
better-suited i-vector representation for a particular speaker can be estimated
from amongst i-vectors extracted from different acoustic features. One way of
doing this is by determining the i-vector representation which has the maximum
distance between the given speaker and the other enrollment speakers. If there
are $N$ speakers, the optimal i-vector representation $\Hat{p}$ 
for the $i$th speaker can be determined as

\begin{equation}
\Hat{p} = \arg\max_p \{s_p\},
\label{eq:opt_ivec}
\end{equation}

where 

\begin{equation}
s_p = \frac{\displaystyle \sum_{j=1, \; i \neq j}^N d(w_{p,i},w_{p,j})}{N}.
\label{eq:sp}
\end{equation}

Here, $w_{p,j}$ represents the enrollment i-vector for the $j$th speaker
extracted using the $p$th feature representation. $d$ is a distance measure (for
example, cosine distance) between i-vectors. For the $i$th speaker, the average
distance with the other enrollment speakers is used to determine the optimal
i-vector representation. 

\section{Features from magnitude and phase spectra}
\label{sec:featExt}

The underlying assumption in feature-switching is that information in different
feature representations can be utilized dynamically. Earlier studies
\cite{complement1}, \cite{complement2}, \cite{complement2} have demonstrated
complementary information in magnitude and phase spectra. Mel frequency cepstra
(MFCC) are derived from the magnitude spectrum. A popular method for utilizing
information from the phase spectrum is via group delay functions
\cite{group_delay}. The modified group delay feature (MODGDF) \cite{modgd_feat},
which is derived from the modified group delay function \cite{modgd_func}, have
been explored as complementary features to MFCCs. The procedure for extracting
MODGDF features are briefly described below. More details regarding the theory
of MODGF can be found in \cite{modgd_func} and \cite{modgd_feat}. The algorithm
for extracting the MODGDF feature is given in algorithm \ref{algo:modgd} as 
in \cite{hegdeModgdf}.

\begin{algorithm}[!th]
\caption{MODGDF feature extraction}
\label{algo:modgd}
\begin{algorithmic}[1]

\STATE \COMMENT{\textbf{Input:} A frame of speech $x(n)$\\
\textbf{Output:} MODGDF features $c(n)$}
\STATE Compute the DFT of the speech frame $x(n)$ as $X(k)$.
\STATE Next, the DFT of the signal $n\,x(n)$ is computed as $\hat{X}(k)$.
\STATE Compute the cepstrally smoothed spectra of $X(k)$ and denote it as
$S(k)$. The parameter $\textrm{lifter}_\omega$ is used to control the length of
the window in the cepstral domain. 
\STATE Compute the MODGD as:
\begin{equation*}
\tau_m(k) = \left(\frac{\tau(k)}{|\tau(k)|}\right) (|\tau(k)|)^\alpha
%\label{eq:modgd}
\end{equation*}
where
\begin{equation*}
\tau(k) = \frac{X_R(k)\hat{X}_R(k)+X_I(k)\hat{X}_I(k)}{|S(k)|^{2\gamma}}
\end{equation*}
and the parameters $\alpha$ and $\gamma$ are used to control the dynamic range
of the MODGD. 
\STATE Compute the MODGDF features by taking the DCT:
\begin{equation*}
c(n) = \sum_{k=0}^{N_f-1}\tau_m(k)\cos(n(2k+1)\pi/N_f) \quad 0 \leq n < N_c
\end{equation*}
where $N_f$ is the DFT size and $N_c$ are the number of cepstral coefficients. 

\end{algorithmic}
\end{algorithm}

\section{Experimental evaluation}
\label{sec:expts}
This section details the experimental evaluation of speaker verification in the
feature-switching framework. We give details about the dataset used, the
development of the feature-switching system, and comparisons with baseline and
fusion systems.

\flushend
\subsection{Dataset description}
\label{sec:dB}

\begin{table*}[t]
\centering
\caption{NIST 2010 conditions used in the evaluation}
\resizebox{0.75\textwidth}{!}{
\begin{tabular}{|c|c|c|c|}
\hline
Condition & Channel & Training vocal effort & Testing vocal effort \\ 
\hline \hline
C5 & Telephone & Normal & Normal \\ \hline
C6 & Telephone & Normal & High \\ \hline
C7 & Microphone & Normal & High \\ \hline
C8 & Telephone & Normal & Low \\ \hline
C9 & Microphone & Normal & Low \\ \hline
\end{tabular}
}
\label{tab:datasetConditions}
\end{table*}

Speaker verification experiments are performed on a subset of
the NIST 2010 SRE dataset. Telephone and microphone utterances under varying 
vocal effort, as detailed in \cite{nist2010SRE} are used for enrollment and 
evaluation. These are summarised in Table \ref{tab:datasetConditions}. 
Gender-specific hyper-parameters for the speaker recognition systems including 
the UBM and the T-matrix are developed using data 
from SRE99, SRE03, SRE04, SRE05, SRE06, SRE08, and SRE08-extended data.


\subsection{Baseline verification systems}
\label{subsec:baseline}

Feature-switching is performed on two speaker verification frameworks: the GMM-UBM system, 
and the i-vector system. The evaluation
metric used is the equal error rate (EER), and is evaluated separately
for male and female genders.


\textbf{Voice activity detection (VAD)}: VAD is an important component in speech
processing systems. In our systems, speech frames of 25 ms size, with a frame
shift of 10 ms are utilized. Since the utterances are fairly clean, a simple VAD
using threshold on average short-term energy is utilized. This typically
discards about 20-25\% of the input frames.

\textbf{GMM-UBM system:} GMM-UBM systems are developed separately for 
MFCC and MODGDF feature representations. 38-dimensional feature vectors are 
extracted in each feature domain. Gender-dependent 1024-mixture UBMs are built 
from development data.  Speaker-dependent GMM models are generated for the 
enrollment data by adapting the means of top 10 maximum contributing mixture 
components of the UBM. For each test utterance, similarity scores are computed 
as the ratio of the log-likelihood of the extracted features with the speaker
model and the UBM.

The baseline systems are denoted as follows. The names of the various systems
compared are self-descriptive, and individual systems are built for each gender.
\begin{enumerate}
\item UBM-MFC: UBM-GMM verification system with MFCC features
\item UBM-MGD: UBM-GMM verification system with MODGD features
\end{enumerate}

\textbf{i-vector system:} 
38-dimensional feature vectors are extracted and first and second order
super-vector statistics are computed using a 1024-mixture UBM. A total
variability matrix of size 38912 $\times$ 600 is randomly initialized and
estimated using development data, as detailed in \cite{dehak_ivector,
kenny_JFA}.  500-dimensional i-vectors are estimated for each enrollment
utterance and test utterance. Preprocessing steps to reduce channel variability
including i-vector length normalization \cite{garciaRomero}, linear discriminant
analysis (LDA) and within-class covariance normalization (WCCN) are applied to
the i-vectors.  The LDA projection matrix is learnt by utilizing
speaker-specific recordings from the development data. %Is this gender specific? - Yes
Cosine similarity between enrollment and test i-vectors is utilized to determine
the similarity score between them.The different baseline systems developed in
this framework are referred as ivec-MFC and ivec-MGD. As before, these are
gender specific.

In both the GMM-UBM and the i-vector frameworks, similarity
scores calculated between the test and enrollment utterances 
are subjected to T-normalization \cite{tnorm}. Mean and variance for T-matrix is
estimated from the target and non-target utterances of each speaker given in the 
NIST SRE 2010 database.
%What is the data used to estimate the mean and variance?
%True and impostor
%models for every speaker is given along with the database. Each 
%test utterance is scored against those fixed set of true and imposter 
%speaker models. From this set of scores, the mean and standard
%deviation are computed and T-Normalization is performed. 
The EERs of these baseline systems are listed in the Table \ref{tab:eer}.
%%% \textbf {What is the devel data for score norm?}


%\textbf{Feature-level fusion:} 
%Feature-level fusion, also known as early fusion (EF),
%is achieved by concatenating the 38-dimensional MFCC and MODGDF feature vectors 
%used in baseline systems to get a 76-dimensional feature vector. Using these
%concatnated feature vectors, gender-dependent speaker verification systems
%are built. These systems are indicated as EF-UBM, and EF-ivec, respectively for
%the GMM-UBM and i-vector frameworks. The EERs of these systems are shown in 
%Table  \ref{tab:eer}.

\textbf{Score-level fusion:}  Score-level fusion, also called late fusion (LF)
is achieved by fusing the scores of individual feature-based baseline systems
\cite{fusion}. The fusion scores are a linear combination of the MFCC and MODGDF
scores. A line search is performed to find the optimal weighing parameter. These
systems are denoted as LF-UBM and LF-ivec.  The performance of the fusion
systems are given in Table \ref{tab:eer}. 

	
\subsection{Feature-switching}
\label{subsec:featSwitch}

In the proposed feature-switching framework, different speaker
claims are verified using different feature representations. 
Experimental evaluation of feature-switching is performed in both
GMM-UBM and i-vector frameworks. The details of these systems
are given below.

\begin{figure}[h]
\includegraphics[scale=0.25]{../figures/FeatureSwithcing.eps}
\caption{Testing phase of feature-switching system}
\label{fig:systemArch2}
\end{figure}


\textbf{GMM-UBM feature-switching system:}
The baseline systems (MFCC and MODGDF) described in section
\ref{subsec:baseline} form the constituent systems for feature-switching. This
is shown in Figure \ref{fig:systemArch2}. For each enrollment speaker, the
optimal feature is determined from the enrollment utterance, as described in
section \ref{subsec:ubm_optFeat}. Here, the number of candidate features, $P$,
is two, with $p=1$ meaning MFCC features and $p=2$ meaning MODGDF features.

The weighting parameter $\alpha$ is used as a trade-off between mutual information and
KL-divergence for determining the optimal feature of a given speaker (equation
\ref{eq:phiFunc}). Different speakers can have different $\alpha$ values. Since
there is no theoretical insight to determining the correct weighting feature,
the following procedure is adopted for each of the $N$ enrollment speakers.
For a given speaker, the value of $\phi_p$ is determined across various values
of $\alpha$ as, 
\begin{equation}
\phi_p = \max \; (\alpha \theta_p + (1-\alpha) \gamma_p),
\end{equation}
where the values of $\alpha$ are varied from $0$ to $1$ in steps of $0.1$ and $\theta$ and 
$\gamma_p$ are defined in Equation \ref{eq:phiFunc} . Once
$\phi_p$ is determined for $p=1$ and $p=2$, the optimal feature is computed as in equation
\ref{eq:ubm_optFeat}.
%Out of 1402 male  and 1758 female enrollment speakers in the SRE 2010
%dataset, 734 male  and 641 female speakers chose MODGDF and 
%remaining chose MFCC as their optimal feature. Approximately, 890
%male utterances and 1100 female utterances were used across the five
%different test conditions listed in Section \ref{sec:dB}. 
The feature-switching system developed in this framework is denoted as
FS-UBM. 
%The distribution of speakers in both feature
%domain for these feature switching systems are listed in the Table
%\ref{tab:ubm-spkrCount}.


\textbf{i-vector feature-switching system:}
As in the case of the GMM-UBM system, the MFCC and MODGDF i-vector systems
described in section \ref{subsec:baseline} are the constituent systems for the
feature-switching system. For each speaker, the optimal feature representation
is computed as described in equation \ref{eq:opt_ivec}. The feature-switching 
system developed in this framework are denoted as FS-ivec. 
The resulting EERs for these systems are also described in Table \ref{tab:eer}.
Note that the optimal feature is determined for a speaker based on the enrollment data
alone and is independent of the testing condition.


\begin{table*}
\centering
\caption{EERs (in \%)for NIST 2010 male and female trials, conditions C5-C9 }

\begin{subtable}{.5\linewidth}
\centering
{
\caption{\small Male trials using UBM-GMM}
\resizebox{0.95\textwidth}{!}{
\begin{tabular}{|c|c|c|c|c|c|}
\hline
{\bf System} & {\bf C5 } &  {\bf C6} & {\bf C7} & {\bf C8}  & {\bf C9  }\\ \hline \hline
%\multicolumn{6}{|c|}{\bf UBM-Framework Male} \\ \hline
UBM-MFC & 17.27 & 18.09 & 17.28 & 9.97 & 4.55 \\ \hline
UBM-MGD & 19.04 & 19.15 & 11.65 & 13.65 & 5.19 \\ \hline
%EF-UBM & 17.97 & 20.22 & 12.84 & 15.94 & 6.14\\ \hline
LF-UBM & 16.71 & 16.38 & 11.04 & 9.79 & 4.43 \\ \hline
FS-UBM & {\bf 10.48} & {\bf 12.32} & {\bf 9.06} & {\bf 9.4} & {\bf 3.21} \\  \hline
\end{tabular}
}
\label{tab:eer_ubm_male}}
\end{subtable}%
\begin{subtable}{.5\linewidth}
\centering
\caption{\small Female trials using UBM-GMM}
{
\resizebox{0.95\textwidth}{!}{
\begin{tabular}{|c|c|c|c|c|c|}
\hline 
{\bf System} & {\bf C5 } &  {\bf C6} & {\bf C7} & {\bf C8}  & {\bf C9  }\\ \hline \hline
%\multicolumn{6}{|c|}{\bf UBM-Framework Female} \\ \hline
UBM-MFC & 18.07 & {\bf 15.04} & 27.43 & 11.01 & 3.74 \\ \hline
UBM-MGD & 18.37 & 22.02 & 17.51 & 14.32 & 6.87 \\ \hline
%EF-UBM & 17.56 & 20.67  & 23.1  & 13.11 & 5.42  \\ \hline
LF-UBM & 17.80 & 20.27 & 17.26 & 10.40 & {\bf 3.24} \\ \hline
FS-UBM & {\bf 15.12 } & {18.39} & {\bf 14.14} & {\bf 9.89} & 4.27 \\ \hline
\end{tabular}
}
\label{tab:eer_ubm_female}}
\end{subtable}

\begin{subtable}{.5\linewidth}
\centering
{
\caption{\small Male trials using i-vector}
\resizebox{0.95\textwidth}{!}{
\begin{tabular}{|c|c|c|c|c|c|}
\hline
{\bf System} & {\bf C5 } &  {\bf C6} & {\bf C7} & {\bf C8}  & {\bf C9  }\\ \hline \hline
%\multicolumn{6}{|c|}{\bf ivec-Framework Male} \\ \hline
ivec-MFC & 5.10 & 6.20 & 7.44 & 2.17 & 3.13 \\ \hline
ivec-MGD &  6.26 & 8.02 & 7.06 & 3.02 & 3.31  \\ \hline
%EF-ivec & 18.61 & 18.15 & 12.7 & 11.1 & 6.41 \\ \hline
LF-ivec & 4.48 & 5.39 & 6.74 & 1.99 & 3.09 \\ \hline
FS-ivec & {\bf 3.56} & {\bf 4.86} & {\bf 5.57} & {\bf 1.36} & {\bf 2.48} \\ \hline
\end{tabular}
}
\label{tab:eer_ivec_male}}
\end{subtable}%
\begin{subtable}{.5\linewidth}
\centering
\caption{\small Female trials using i-vector}
{
\resizebox{0.95\textwidth}{!}{
\begin{tabular}{|c|c|c|c|c|c|}
\hline 
{\bf System} & {\bf C5 } &  {\bf C6} & {\bf C7} & {\bf C8}  & {\bf C9  }\\ \hline \hline
%\multicolumn{6}{|c|}{\bf ivec-Framework Female} \\ \hline
{ivec-MFC} & 6.25 & 9.32 & 16.55 & 4.60 & 3.65 \\ \hline
{ivec-MGD} & 6.95 & 9.89 & 11.64 & 3.98 & 3.39  \\ \hline 
%{EF-ivec} & 7.91 & 11.8 & 14.83 & 6.13 & 4.64 \\ \hline 
{LF-ivec} & 5.52 & 7.53 & 13.33 & 3.83 & 3.44 \\ \hline
{FS-ivec} & {\bf 4.62} & {\bf 6.58} & {\bf 11.56} & {\bf 3.61} & {\bf 2.92} \\ \hline
\end{tabular}
}
\label{tab:eer_ivec_female}}
\end{subtable}
\label{tab:eer}
\end{table*}


\subsection{Result analysis}
\label{subsec:resAnalysis}

The various speaker verification systems described in Section \ref{sec:expts} 
include four baseline systems (UBM-MFC, UBM-MGD, ivec-MFC, ivec-MGD), 
%two early fusion systems (EF-UBM, EF-ivec), 
two late fusion systems (LF-UBM, LF-ivec) 
and two proposed feature-switching systems (FS-UBM, FS-ivec). Each of these have 
corresponding systems for male trials and female trials. The performance metric 
is the equal error rate (EER) \cite{eer1}. The performance is tabulated in Table \ref{tab:eer}.
Compared to the best baseline systems, the score-fusion systems provide an average
relative improvement of 0.636\% for male trials, and -1.07\% for female trials in the
GMM-UBM case and 0.394\% and 0.186\% for the i-vector case. More specifically, 
for male trials in i-vector case, the condition-wise improvements given by the score-fusion 
systems  with respect to best baseline systems are: C5: 0.62\%, C6: 0.81\%, C7: 0.32\%, 
C8: 0.18\% and C9: 0.04\%. Similarly the condition-wise improvements for
female trials are C5: 0.73\%, C6: 1.79\%, C7: -1.69\%, C8: 0.15\% and C9: -0.05\%.

The proposed feature switching system outperforms the score fusion systems in
almost all the NIST conditions. Compared to the best baseline systems, the average relative
improvement is 3.41\% for male trials and 2.48\% for female trials in UBM-GMM case and 
1.67\% and 1.11\% in i-vector case. The only exception is in condition C6 
in the GMM-UBM case, for female trials, where the baseline
MFCC system achieves the lowest EER.

To understand why feature-switching brings about the improvements, we
analysed the trials in the various evaluation conditions. In the baseline
systems, each evaluation trial gets verified in the same feature space (be it
the GMM-UBM case or the i-vector case.) Whereas in feature-switching, the trials
get evaluated in the optimal feature space of the claimed speaker. This is
summarised in Tables \ref{tab:gmmAnalysis} and \ref{tab:ivecAnalysis}.  In Table
\ref{tab:gmmAnalysis}, the first entry states that the best baseline EER of
17.27\% is achieved when all the trials are evaluated in MFCC space. But in
feature-switching, based on the optimal feature of the claimed speaker, 10131
trials were evaluated in MFCC space and 3934 trials were evaluated in MODGDF
space. This differential evaluation resulted in more number of correct
verifications, hence bringing down the EER. The other entries in Tables
\ref{tab:gmmAnalysis} and \ref{tab:ivecAnalysis} give details of the other
cases. Figure~\ref{fig:scoreDist} shows the score distribution comparison of true and 
impostor trials for condition C5, for the baseline systems and feature switching system 
for male trials in i-vector case.

\begin{table}[h]
\centering
\caption{Distribution of the speakers to MFCC and MODGDF optimal feature spaces.}
\begin{tabular}{|l|l|l|l|}
\hline
Gender & MFCC & MODGDF & Total \\
\hline
\multicolumn{4}{|c|}{FS-UBM system}\\
\hline
Male & 668 (48\%) & 734 (52\%) & 1402 \\
Female & 1117 (64\%) & 641 (36\%) & 1758 \\
\hline
\multicolumn{4}{|c|}{FS-ivec system}\\
\hline
Male & 1376 (98\%)& 26 (2\%)& 1402 \\
Female & 85 (5\%) & 1673 (95\%) & 1758 \\
\hline
\end{tabular}
\label{tab:optFeat}
\end{table}

\begin{table*}[h!tb]
\centering
\caption{The distribution of speaker trials in the feature spaces using feature switching 
in UBM-GMM framework. (EERs of baseline system and feature switching systems are 
compared and lesser EER in given in bold font)}
\begin{tabular}{cc|c|c|c|c|}
\cline{3-6}
\multicolumn{1}{l}{} & \multicolumn{1}{l|}{} & \multicolumn{4}{c|}{UBM-GMM Framework} \\ \hline
\multicolumn{1}{|c|}{Gender} & Condition & \begin{tabular}[c]{@{}c@{}}No. of \\ trials \end{tabular} & \begin{tabular}[c]{@{}c@{}}Best Baseline EER \\  and Feature
Space\end{tabular} & \begin{tabular}[c]{@{}c@{}}Feature \\ Switching\\ EER\end{tabular} &
  \begin{tabular}[c]{@{}c@{}}No. of trails\\ evaluated in \\ MGD/MFCC (MGD\%/MFCC\%)\end{tabular} \\ \hline
\multicolumn{1}{|c|}{\multirow{5}{*}{Male}} & C5 & 14065 & 17.27 (MFC) & {\bf 10.48} & 3934/10131 (27.97/72.03) \\ \cline{2-6} 
\multicolumn{1}{|c|}{} & C6 & 12975 & 18.09 (MFC) & {\bf 12.32} & 2904/10071 (22.38/77.62) \\ \cline{2-6} 
\multicolumn{1}{|c|}{} & C7 & 12938 & 11.65 (MGD) & {\bf 9.06} & 3915/9023 (30.26/69.74)\\ \cline{2-6} 
\multicolumn{1}{|c|}{} & C8 & 11116 & 9.97 (MFC) & {\bf 9.4} & 2193/8923 (19.73/80.27)\\ \cline{2-6} 
\multicolumn{1}{|c|}{} & C9 & 10815 & 4.55 (MFC) & {\bf 3.21} & 3281/7534 (30.34/69.66)\\ \hline
\multicolumn{1}{|c|}{\multirow{5}{*}{Female}} & C5 & 16317 & 18.07 (MFC) & {\bf 15.12} & 2501/13817 (15.32/84.68)\\ \cline{2-6} 
\multicolumn{1}{|c|}{} & C6 & 15673 & {\bf 15.04} (MGD) & {18.39} & 2400/13273 (15.31/84.69)\\ \cline{2-6} 
\multicolumn{1}{|c|}{} & C7 & 15398 & 17.51 (MGD) & {\bf 14.14} & 4542/10856 (29.50/70.50)\\ \cline{2-6} 
\multicolumn{1}{|c|}{} & C8 & 17495 & 11.01 (MFC) & {\bf 9.89} & 2747/14748 (15.70/84.30) \\ \cline{2-6} 
\multicolumn{1}{|c|}{} & C9 & 16716 & {\bf 3.74} (MFC) & {4.27} & 4880/11836 (29.19/70.81)\\ \hline
\end{tabular}
\label{tab:gmmAnalysis}
\end{table*}

\begin{table*}[h!tb]
\centering
\caption{The distribution of speaker trials in the feature spaces using feature 
switching in i-vector framework. (EERs of baseline system and feature switching 
systems are compared and lesser EER in given in bold font)}
\begin{tabular}{cc|c|c|c|c|}
\cline{3-6}
\multicolumn{1}{l}{} & \multicolumn{1}{l|}{} & \multicolumn{4}{c|}{i-vector Framework} \\ \hline
\multicolumn{1}{|c|}{Gender} & Condition & \begin{tabular}[c]{@{}c@{}}No. of \\ trails
\end{tabular} & \begin{tabular}[c]{@{}c@{}}Best Baseline EER \\  and Feature
Space\end{tabular} & \begin{tabular}[c]{@{}c@{}}Feature \\ Switching\\ EER\end{tabular} &
  \begin{tabular}[c]{@{}c@{}}No. of trails\\ evaluated in \\ MGD/MFCC(MGD\%/MFCC\%)\end{tabular} \\ \hline
\multicolumn{1}{|c|}{\multirow{5}{*}{Male}} & C5 & 14065 & 5.10 (MFC) & {\bf 3.56} & 376/13698 (2.67/97.33)
\\ \cline{2-6} 
\multicolumn{1}{|c|}{} & C6 & 12975 & 6.20 (MFC) & {\bf 4.86} & 89/12886 (0.69/99.31) \\ \cline{2-6} 
\multicolumn{1}{|c|}{} & C7 & 12938 & 7.06 (MGD) & {\bf 5.57} & 142/12796 (1.09/98.91) \\ \cline{2-6} 
\multicolumn{1}{|c|}{} & C8 & 11116 & 2.17 (MFC) & {\bf 1.36} & 97/11019 (0.87/99.13) \\ \cline{2-6} 
\multicolumn{1}{|c|}{} & C9 & 10815 & 3.13 (MFC) & {\bf 2.48} & 45/10770 (0.42/99.58) \\ \hline
\multicolumn{1}{|c|}{\multirow{5}{*}{Female}} & C5 & 16317 & 6.25 (MFC) & {\bf 4.62} & 15233/1084 (93.36/6.64)
\\ \cline{2-6} 
\multicolumn{1}{|c|}{} & C6 & 15673 & 9.32 (MFC) & {\bf 6.58} & 15363/309 (98.02/1.98) \\ \cline{2-6} 
\multicolumn{1}{|c|}{} & C7 & 15398 & 11.64 (MGD) & {\bf 11.56} & 14772/626 (95.93/4.07) \\ \cline{2-6} 
\multicolumn{1}{|c|}{} & C8 & 17495 & 3.98 (MGD) & {\bf 3.61} & 17199/296 (98.31/1.69) \\ \cline{2-6} 
\multicolumn{1}{|c|}{} & C9 & 16716 & 3.39 (MGD) & {\bf 2.92} & 16041/675 (95.96/4.04) \\ \hline
\end{tabular}
\label{tab:ivecAnalysis}
\end{table*}

\begin{figure*}[h!tb]
\centering
\includegraphics[width=0.75\textwidth]{../figures/m-det5-scoresDist.eps}
\caption{Score distribution of baseline systems (MFCC and MODGD) and feature switching (FS) system for male database in test condition C5 using i-vector case.}
\label{fig:scoreDist}
\end{figure*}


%%MODGDF is known to capture the higher formants clearly. In i-vector framework, for female trials, MODGDF is chosen as the optimal feature. This may be due to the richness of higher formants in female voice. 


The distribution of enrollment speakers among the optimal feature spaces
is given in Table \ref{tab:optFeat}. The optimal feature of a given speaker need
not be the same for the UBM-GMM feature-switching system and the i-vector
feature-switching system. %This contradictory behaviour may be due to the LDA and WCCN performed in i-vector case. When WCCN is performed, the variance becomes uniform in all the directions (Figure \ref{fig:ubmivec}) and LDA chooses the direction that maximises the distance between the centroids.  LDA and WCCN makes i-vector to behave more like a discriminative classifier whereas UBM-GMM behaves like a codebook. 
In the UBM-GMM feature switching system, male speakers and female speakers are more or less
evenly split in the different feature spaces. Whereas, in the i-vector case, most of the male 
speakers get MFCC as their optimal feature and females get MODGDF. These two techniques are 
based on different principles, and hence it is difficult to compare them directly. Steps like LDA and 
WCCN may provide more discriminative abilities to the i-vector framework.

%\begin{figure}[h!tb]
%\centering
%\includegraphics[width=0.5\textwidth]{../figures/UBMvsIVec.eps}
%\caption{Projection in UBM-GMM and i-vector case}
%\label{fig:ubmivec}
%\end{figure}

\section{Conclusion}
\label{sec:conclude}

In this paper, we developed the paradigm of feature-switching to perform
text-independent speaker verification. By performing verification in a
feature space that is well-suited to the speaker under consideration,
improvements in accuracy are obtained. The method is evaluated using the
classical GMM-UBM speaker verification framework, as well as the i-vector
framework. Once the well-suited feature representation of a given speaker is
determined, verification of that speaker can be performed in that feature space.
Our experimental evaluation demonstrates that feature-switching
provides benefits above that of conventional system fusion. On the NIST 2010 SRE
dataset, an average improvement of 2.062\% and 1.138\% is attained UBM-GMM and i-vector case respectively.

In principle, the method of feature-switching can be applied to any verification
task; for example, to face verification. In this paper, we have applied
feature-switching between two feature representations, which are derived
respectively from magnitude and phase, and are known to be complementary. 
The number of feature representations can be larger. Future
research directions can include a more robust procedure to determine the optimal
feature for a given speaker. An extended version of this might be to have customized
feature representations for every class under consideration, and
feature-switching between them during verification.

\section{Acknowledgement}
\label{sec:ack}
The authors would like to acknowledge the Defence Research and Development 
Organization, India for funding the research under the project 
CSE1314142DRDOHEMA. 

\flushend
%% The Appendices part is started with the command \appendix;
%% appendix sections are then done as normal sections
\appendix


\section*{References}
\bibliographystyle{elsarticle-num}
%\bibliography{refs1}
\bibliography{science}

\end{document}


